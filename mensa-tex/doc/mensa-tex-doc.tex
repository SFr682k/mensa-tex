%%	This is file 'mensa-tex-doc.tex', Version 2017-09-21
%%	Copyright 2017 Sebastian Friedl <sfr682k@t-online.de>
%% 
%%	This work may be distributed and/or modified under the conditions of the LaTeX Project
%%	Public License, either version 1.3c of this license or (at your option) any later version.
%%	The latest version of this license is available at
%%		http://www.latex-project.org/lppl.txt
%%	and version 1.3c or later is part of all distributions of LaTeX version 2008-05-04 or later.
%%
%%	This work has the LPPL maintenace status 'maintained'.
%%	Author: Sebastian Friedl
%%	Current maintainer of this work is Sebastian Friedl
%%
%%	This work consists of the files mensa-tex.cls and mensa-tex-doc.tex
%%
%%	---------------------------------------------------------------------------------------------------------------------------------------------
%%
%%	A LaTeX class for typesetting school cafeteria menus consisting of two lunches (with dessert) and dinner
%%
%%	---------------------------------------------------------------------------------------------------------------------------------------------
%%
%%	Please report bugs and other problems as well as suggestions for improvements to the following email address: sfr682k@t-online.de
%%
%%	--------------------------------------------------------------------------------------------------------------------------------------------- 


% !TeX spellcheck=en_GB
% !TeX program=lualatex



\documentclass[11pt]{ltxdoc}

\usepackage[utopia]{mathdesign}
\usepackage[no-math]{fontspec}
\usepackage{polyglossia}
\setdefaultlanguage{english}

\usepackage{array}
\usepackage{csquotes}
\usepackage{hyperref}
\usepackage{multicol}
\usepackage[english]{selnolig}

\parindent 0pt

\setmainfont[Numbers=OldStyle]{erewhon}
\setsansfont[Numbers=OldStyle,Scale=MatchLowercase]{Source Sans Pro}
\setmonofont[Numbers=OldStyle,Scale=MatchLowercase]{Source Code Pro}

\usepackage[left=4.50cm,right=2.75cm,top=3.25cm,bottom=2.75cm,nohead]{geometry}

\hyphenation{}

\title{The \texttt{mensa-tex} class \\ {\large\url{https://github.com/SFr682k/mensa-tex}}}
\author{Sebastian Friedl \\ \href{mailto:sfr682k@t-online.de}{\ttfamily sfr682k@t-online.de}}
\date{2017/09/21}

\hypersetup{pdftitle={The mensa-tex class},pdfauthor={Sebastian Friedl}}

\begin{document}
	\maketitle
	\thispagestyle{empty}
	
	\begin{center} \itshape
		\enquote{I can't go to a restaurant and order food \\ because I keep looking at the fonts on the menu} \\
		--- \textsc{\upshape Donald E. Knuth} ---
	\end{center}
	
	\medskip
	\begin{abstract}
		\noindent%
		A \LaTeX\ class for typesetting school cafeteria menus consisting of two lunches (with dessert) and dinner
	\end{abstract}
	
	
	\tableofcontents
	
	\clearpage
	
	

	\subsection*{Dependencies and other requirements}
	\addcontentsline{toc}{subsection}{Dependencies and other requirements}
	The \texttt{mensa-tex} class requires \LaTeXe\ and the following packages:
	\begin{multicols}{3}\ttfamily\centering
		array \\ colortbl \\ datetime2 \\ datetime2-calc \\ geometry \\ graphicx \\ lmodern \\ textcomp \\ xcolor
	\end{multicols}
	
	
	\subsection*{License}
	\begin{small}
		\addcontentsline{toc}{subsection}{License}
		\textcopyright\ 2017 Sebastian Friedl
		
		\smallskip
		This work may be distributed and/or modified under the conditions of the \LaTeX\ Project Public License, either version 1.3c of this license or (at your option) any later version.
		
		\smallskip
		The latest version of this license is available at \url{http://www.latex-project.org/lppl.txt} and version 1.3c or later is part of all distributions of \LaTeX\ version 2008-05-04 or later.
		
		\smallskip
		This work has the LPPL maintenace status \enquote*{maintained}. The current maintainer of this work is Sebastian Friedl. \\
		This work consists of the following files:
		\begin{itemize} \itemsep 0pt
			\item \texttt{mensa-tex.cls} and
			\item \texttt{mensa-tex-doc.tex}
		\end{itemize}
	\end{small}


	\subsection*{Call for cooperation}
	\addcontentsline{toc}{subsection}{Call for cooperation}
	Please report bugs and other problems as well as suggestions for improvements using the \href{https://github.com/SFr682k/mensa-tex/issues}{issue tracker on GitHub} or sending an email (\href{mailto:sfr682k@t-online.de}{\texttt{sfr682k@t-online.de}}).


	\clearpage
	
		
		
	% DOCUMENTATION PART ----------------------------------------------------------------------
	
	\section{Using the \texttt{\textbackslash documentclass} command}
	Using this class is as easy as using the \verb|\documentclass{mensa-tex}| command.
	
	\bigskip
	Following class options are available:
	\begin{itemize}
		\item[\texttt{app}] Use a layout optimized for small screens using DIN/ISO A6 paper
		\item[\texttt{en-GB}] Use an English localisation, British variant \textit{(default)}
		\item[\texttt{en-US}] Use an English localisation, American variant
		\item[\texttt{german}] Use a German localisation
	\end{itemize}


	\section{Creating a menu}
	The following section deals with creating a menu using \texttt{mensa-tex}. \\
	All the \textit{commands are to be used inside the preamble} since the menu gets created instantly when using \verb|\begin{document}|.
	
	\subsection{Setting up the basic information}
	The basic information consists of the name of the cafeteria, the institute (or school) it is located at and the image used on the single pages. \\
	It can be set by using the following commands:
	
	\medskip
	\DescribeMacro{\mensaname}
	This command is used to declare the name of the cafeteria -- maybe something like \verb|\mensaname{Food Corner}| \textit{(default is the plain old boring \enquote{Mensa})}. \\
	You may want to change the font size by using arbitrary \LaTeX\ font size commands.
	
	\medskip
	\DescribeMacro{\institute}
	Sets the name of the institute the cafeteria is located at -- for example, you can insert \verb|\institute{University of LOL}| in your preamble if your cafeteria is located at some institute called the \enquote{University of LOL}. \\
	Note that information about the institute is only printed when using the normal layout.

	\medskip
	\DescribeMacro{\setimage}
	Add an image to your diet plans using this command. \\
	Note that you \textit{have} to declare the image by using the \verb|\includegraphics| inside \verb|\setimage| (e.~g.~\verb|\setimage{\includegraphics[width=\linewidth]{path/to/picture}}|). \\
	The space available for the image depends on the used layout (see table \ref{tab:image-sizes}).
	
	\begin{table}\centering\sffamily\small\renewcommand{\arraystretch}{1.25}
		\begin{tabular}{r*{2}{|>{\centering\ttfamily}m{.25\textwidth}<{\arraybackslash}}}
			       & \textsf{normal layout}                                & app \textsf{layout}                                   \tabularnewline\hline\hline
			 width & \verb|.50\textwidth|  \newline $\approx$ \verb|9.0cm| & \verb|.58\textwidth|  \newline $\approx$ \verb|4.9cm| \tabularnewline\hline
			height & \verb|.15\textheight| \newline $\approx$ \verb|4.0cm| & \verb|.15\textheight| \newline $\approx$ \verb|1.9cm|
		\end{tabular}
	
		\rmfamily
		\caption{Available space for the header image}
		\label{tab:image-sizes}
	\end{table}
	
	
	\subsection{Adding food}
	For adding information about the food, the following six commands are provided.
	
	\medskip
	\DescribeMacro{\startdate}
	This commands defines the \enquote{start date}, the date of the first entry \textit{(the start date's weekday happens to be a Monday)}.	The start date has to be present in \texttt{YYYY-MM-DD} format, e.~g.~\verb|\startdate{2007-01-01}| \textit{(Default is 2001-01-01)}.
	
	\medskip
	\DescribeMacro{\monday}\DescribeMacro{\tuesday}\DescribeMacro{\wednesday}\DescribeMacro{\thursday}\DescribeMacro{\friday}
	Use these five commands to insert food into the empty diet. \\
	Every single command requires the same four arguments:
	\begin{multicols}{4}
		\begin{enumerate}\itemsep0pt
			\item Menu I
			\item Menu II
			\item Dessert
			\item Lunch
		\end{enumerate}
	\end{multicols}

	For example, to obtain Monday's menu consisting of
	\begin{multicols}{2}
		\begin{tabbing}
			\hspace{2cm}\=\kill
			Menu I:		\> Fish and chips \\
			Menu II:	\> Crispy fried chicken \\
			Dessert:	\> Chocolate fudge \\
			Lunch:		\> DIY hamburgers
		\end{tabbing}
	\end{multicols}
	you have to write \\
	\verb|\monday{Fish and chips}%          <-- % is required for line breaks| \\\nopagebreak[4]
	\verb|       {Crispy fried chicken}%| \\\nopagebreak[4]
	\verb|       {Chocolate fudge}%| \\\nopagebreak[4]
	\verb|       {DIY hamburgers}|
	
	\smallskip
	It is possible to insert the command listed above without line breaks, however, doing so will result in the source being less human--readable.
	
	\smallskip
	Due to the menu being implemented in a \texttt{tabular} environment, you have to use \verb|\newline| instead of \,\verb|\\|\, to produce additional lines. \\
	Possible, additional hyphenations not found by \LaTeX\ can be marked by manually inserting discretionary hyphens (\verb|\-|) (e.~g.~\verb|hyphen\-ation|).
	
	
	
	% TODO: Ab hier weiterschreiben
	\subsection{Adding additional information}
	% \longremarks
	% \shortremarks
	
	\subsection{Using fancy colours}
	% \setbgcolor
	% \setcolorfg
	% \setctextcolor
	
	% \sup
	% \vgt	\vgn
	
	
	\section{Adding support for other languages}
	
	
	% EXAMPLES
\end{document}